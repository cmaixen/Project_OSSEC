\chapter{Introduction}\label{introduction}

In this document we will the describe the mini-project in assignment of the course Operating Systems and Security (OSSEC) at the Vrij Universiteit Brussel. For this mini-project we dig deeper in the world of malware and more specific ransomeware. This document is divided in two chapters. In chapter \ref{Ransomeware} we talk about the analysis of the malware. First we explain in \ref{Definition} the terminology to get familiar with the subject. Next in section \ref{workflow} we look at the main behavior of and the used vulnerabilities by the ransomeware. As the  last section of this chapter we look at the past generations of ransomeware to get a clear idea how it is working and how it is evolved. In chapter \ref{Creation of the Ransomeware} we implement our own ransomeware, where we see in every section a required buildingblock to implement the ransomeware. The ransomeware will be development for Windows 10. This is the newest operating systems by Microsoft, where Microsoft mainly forces his users to update to this operation systems. This is why it is highly interesting to develop a ransomeware for this platform.

