\chapter{Introduction}\label{introduction}

In this document we will the describe the mini-project in assignment of the course Operating Systems and Security (OSSEC) at the Vrij Universiteit Brussel. For this mini-project we implemented a self-written ransomeware. This document is divided in three chapters. Chapter \ref{Ransomeware} contains lecture about the ransomeware. It is necessary to have first a clear understanding of how ransomware works before we start explaining the code in chapter \ref{Creation of the ransomware}. The first chapter is divided in two parts. First in section \ref{Definition} we explain the terminology to get familiar with the subject and in the second part we discus the general workflow of ransomeware. In chapter \ref{Creation of the Ransomeware} we eventually discus the self-written ransomeware, where we highlight in every section a specific functionality. In the last chapter we do a small analysis of our ransomeware concerning the recoverability of the files and the detection rate by virus scanners.The ransomeware is developed for and tested on Windows 7. This is still the most used operation systems, with more than 44,5\% \cite{OSstats}. This is why it is highly interesting to develop a ransomeware for this platform.

